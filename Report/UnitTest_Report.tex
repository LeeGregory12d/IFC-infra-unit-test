\documentclass{scrartcl}

\usepackage[utf8]{inputenc}

\usepackage{amsmath}
\usepackage{babel}

\usepackage{xcolor}
\usepackage{tikz}
\usetikzlibrary{arrows.meta,calc,decorations.markings,arrows,positioning,fit,intersections,patterns,scopes,datavisualization,backgrounds}
\usetikzlibrary{shapes.geometric,decorations.pathreplacing,shadings,decorations.text}

\usepackage{hyperref}
\usepackage[footnotes,definitionLists,smartEllipses,hybrid,pipeTables=true,shiftHeadings=1,tableCaptions=true]{markdown}
\usepackage[edges]{forest}

\usepackage{cleveref}

\title{Appendix A: Unit Test Documentation}
\author{InfraRoom: IFC Infrastructure Extension Deployment project}
\date{document automatically generated on: \today}

\setkeys{Gin}{width=\linewidth}

\usepackage{ltablex}
\usepackage{tabularx}
\usepackage{booktabs}

\makeatletter
\def\markdownLaTeXReadAlignments#1{%
\advance\markdownLaTeXColumnCounter by 1\relax%
\if#1d%
\addto@hook\markdownLaTeXTableAlignment{p{0.5\textwidth}}%
\else%
\addto@hook\markdownLaTeXTableAlignment{#1}%
\fi\ifnum\markdownLaTeXColumnCounter<\markdownLaTeXColumnTotal\relax\else\expandafter\@gobble%
\fi\markdownLaTeXReadAlignments%
}
\makeatother

\begin{document}

\maketitle
\addtocontents{toc}{~\hfill\textbf{Page}\par}

\section{Summary}
\label{sec:summary}

This is a summary document of all approved unit tests submitted during the IFC Infrastructure Extensions Deployment project.

This document has been automatically produced from the \texttt{readme} files of individual unit tests.
The originating documentation together with all mentioned files can be obtained
 from the project's official GitHub repository\footnote{\url{https://github.com/bSI-InfraRoom/IFC-infra-unit-test}}.

\subsection{Coverage}
\label{sec:coverage}

\Cref{tab:coverage} lists all IFC entities occurring in all listed unit tests.
The number next to each entity shows how many times this entity occurs.

\begin{tabularx}{\textwidth}{rX}
  \caption{The covered entities of all unit test files combined.}\label{tab:coverage} \\
  \toprule
    Occurance & Entity \\
  \midrule
  \endfirsthead
  \caption[]{\emph{(cont.)} The covered entities of all unit test files combined.} \\
  \toprule
    Occurance & Entity \\
  \midrule
  \endhead
  \bottomrule
  \multicolumn{2}{r}{\footnotesize(continued)}
  \endfoot
  \bottomrule
  \endlastfoot
  \input{../Report/coverageTable.tex}
\end{tabularx}

\subsection{Schematics}
\label{sec:schematics}

\Cref{fig:schematics} represents the dependencies between the individual unit tests.
Unit tests are represented as boxes, while the arrows point from the dependant to the independent unit test.
Each unit test has a link to the corresponding section of this document in the upper right corner.

\tikzset{every label/.style={xshift=3ex, text width=10ex, align=left, 
                             inner sep=1pt, font=\footnotesize}}
\tikzset{pr/.style={dashed}}
\tikzset{star/.style={draw=none,anchor=north east,inner sep=2pt}}
\begin{figure}
  \begin{forest}
    for tree={              % style of tree nodes
      font=\footnotesize,
      draw, semithick, rounded corners,minimum width=7em,
            align = center,
        inner sep = 1pt,
                            % style of tree (edges, distances, direction)
             edge = {draw, semithick, latex'-},
    parent anchor = east,
     child anchor = west,
             grow = east,
    forked edge,            % for forked edge
            l sep = 6mm,   % level distance
         fork sep = 3mm,    % distance from parent to branching point
                }
    [Setup-1,name=boiler1,label=\labelcref{sec:project_setup_1}
      [Road-6,name=road6,label=\labelcref{sec:road_6},pr
      ],
      [Road-2,name=road2,label=\labelcref{sec:road_2},pr
      ],
      [Road-1,name=road1,label=\labelcref{sec:road_1},pr
      ],
      [Spatial-1,name=spatial1,label=\labelcref{sec:spatial_1}
        [Spatial-8,name=spatial8,label=\labelcref{sec:spatial_8},pr
        ],
        [Spatial-7,name=spatial7,label=\labelcref{sec:spatial_7},pr
        ] { \node[star] at (.north east) {*}; },
        [Spatial-6,name=spatial6,label=\labelcref{sec:spatial_6}
        ] { \node[star] at (.north east) {*}; },
        [Spatial-5,name=spatial5,label=\labelcref{sec:spatial_5}
        ],
        [Spatial-4,name=spatial4,label=\labelcref{sec:spatial_4}
          [Drainage-2,name=drainage2,label=\labelcref{sec:drainage_2}
          ] { \node[star] at (.north east) {*}; },
          [Drainage-1,name=drainage1,label=\labelcref{sec:drainage_1}
          ] { \node[star] at (.north east) {*}; },
        ] { \node[star] at (.north east) {*}; },
        [Spatial-3,name=spatial3,label=\labelcref{sec:spatial_3}
          [Other-1,name=other1,label=\labelcref{sec:other_1},pr
          ] { \node[star] at (.north east) {*}; },
        ],
        [Spatial-2,name=spatial2,label=\labelcref{sec:spatial_2}
        ],
      ],
      [Bridge-1,name=bridge1,label=\labelcref{sec:bridge_1},pr
      ],
      [Marine-1,name=marine1,label=\labelcref{sec:marine_1},pr
      ] { \node[star] at (.north east) {*}; },
      [Marine-2,name=marine2,label=\labelcref{sec:marine_2},pr
      ] { \node[star] at (.north east) {*}; },
      [Setup-2,name=boiler2,label=\labelcref{sec:project_setup_2}
      ],
      [GeoRef-1,name=georef1,label=\labelcref{sec:georeferencing_1}
        [Survey-1,name=survey1,label=\labelcref{sec:survey_1},pr
          [Survey-2,name=survey2,label=\labelcref{sec:survey_2},pr
            [Survey-3,name=survey3,label=\labelcref{sec:survey_3},pr
            ],
          ],
        ],
        [GeoRef-3,name=georef3,label=\labelcref{sec:georeferencing_3},pr          
          [GeoRef-4,name=georef4,label=\labelcref{sec:georeferencing_4},pr
          ]
        ],
        [GeoRef-2,name=georef2,label=\labelcref{sec:georeferencing_2},pr
        ],
      ] {
        \draw[-latex'] () to[out=north,in=south] (tin1);
      },
      [TIN-1,name=tin1,label=\labelcref{sec:tin_1}
        [GeoRef-TIN-1,name=georeftin1,label=\labelcref{sec:georef_tin_1} 
          [GeoRef-TIN-4,name=georeftin4,label=\labelcref{sec:georef_tin_4},pr
          ],
          [GeoRef-TIN-3,name=georeftin3,label=\labelcref{sec:georef_tin_3},pr
          ],      
          [GeoRef-TIN-2,name=georeftin2,label=\labelcref{sec:georef_tin_2}
          ],
          [Combi-2,name=combi2,label=\labelcref{sec:georef_earth_1}
            [Combi-3,name=combi3,label=\labelcref{sec:georef_earth_2},pr
            ]
          ] { \draw[-latex'] () to[out=north west,in=east] (earth1); }
        ] { \draw[-latex'] () to[out=south west,in=north east] (georef1); }
      ],
      [Earthworks-3,name=earth3,label=\labelcref{sec:earth_3},pr
      ],
      [Earthworks-0,name=earth0,label=\labelcref{sec:earth_0}
        [Earthworks-2,name=earth2,label=\labelcref{sec:earth_2}
        ],
        [Earthworks-1,name=earth1,label=\labelcref{sec:earth_1}
        ],
      ],
      [Geotechnics-1,name=geotech1,label=\labelcref{sec:borehole_1}
      ] { \node[star] at (.north east) {*}; },      
      [Alignment-17,name=align17,label=\labelcref{sec:align_17},pr  
        [Alignment-18,name=align18,label=\labelcref{sec:align_18},pr
        ],
      ],   
      [Alignment-14,name=align14,label=\labelcref{sec:align_14},pr
        [Road-4,name=road4,label=\labelcref{sec:road_4},pr
        ] { \node[star] at (.north east) {*}; },
        [Road-5,name=road5,label=\labelcref{sec:road_5},pr
        ] { \node[star] at (.north east) {*}; },
      ] { \node[star] at (.north east) {*}; },
      [Alignment-13,name=align13,label=\labelcref{sec:align_13},pr
        [Road-3,name=road3,label=\labelcref{sec:road_3},pr
        ],
      ],
      [Alignment-12,name=align12,label=\labelcref{sec:align_12},pr
        [Alignment-15,name=align15,label=\labelcref{sec:align_15},pr
        ],
        [Alignment-16,name=align16,label=\labelcref{sec:align_16},pr
        ],
      ],
      [Alignment-1,name=align1,label=\labelcref{sec:align_1}
        [Alignment-9,name=align9,label=\labelcref{sec:align_9},pr
          [Alignment-10,name=align10,label=\labelcref{sec:align_10},pr
          ]
        ],
        [Alignment-8,name=align8,label=\labelcref{sec:align_8},pr
        ],
        [Alignment-6,name=align6,label=\labelcref{sec:align_6},pr
          [Alignment-7,name=align7,label=\labelcref{sec:align_7},pr
          ]
        ],
        [Alignment-4,name=align4,label=\labelcref{sec:align_4}
          [Alignment-5,name=align5,label=\labelcref{sec:align_5}
          ],
        ],
        [Alignment-2,name=align2,label=\labelcref{sec:align_2}
          [Alignment-3,name=align3,label=\labelcref{sec:align_3}
          ],
        ],
      ],
    ]  
  \end{forest}
  \caption{Schematic representation of unit tests and their dependencies.
           Dashed boxes mean that they are still under review.
           A star in the upper right corner means that the unit test has been derived from a storyline.}
  \label{fig:schematics}
\end{figure}

\subsection{How to read the document}

Each unit test is within its own section.
The included figures are \emph{floating} in this document, so they may appear at the top of the page,
 even though they are mentioned later in the text.
All figures are always flushed before the next section starts, so that there is no doubt.

\subsection{Overview over the Content}
\renewcommand\contentsname{}
\tableofcontents
\clearpage

% -------------------------------------------------------------------------------
\section{Project Setup}

\subsection{Project Setup 1}
\label{sec:project_setup_1}
\markdownInput{../ProjectSetup-1/readme.md}
\clearpage

\subsection{Project Setup 2}
\label{sec:project_setup_2}
\markdownInput{../ProjectSetup-2/readme.md}
\clearpage

% -------------------------------------------------------------------------------
\section{Georeferencing and Digital Terrain Models (DTMs)}

\subsection{TIN 1}
\label{sec:tin_1}
\markdownInput{../Tin-1/readme.md}
\clearpage

\subsection{Georeferencing 1}
\label{sec:georeferencing_1}
\markdownInput{../Georeferencing-1/readme.md}
\clearpage

\subsection{Georeferencing 2} % PR#85
\label{sec:georeferencing_3}
Under review.%\markdownInput{../Georeferencing-3/readme.md}
\clearpage

\subsection{Georeferencing 3} % PR#85
\label{sec:georeferencing_4}
Under review.%\markdownInput{../Georeferencing-4/readme.md}
\clearpage

\subsection{Georeferencing and DTM 1}
\label{sec:georef_tin_1}
\markdownInput{../Georeferencing-Tin-12d-1/readme.md}
\clearpage

\subsection{Georeferencing and DTM 2}
\label{sec:georef_tin_2}
\markdownInput{../Georeferencing-Tin-12d-2/readme.md}
\clearpage

\subsection{Georeferencing and DTM 3} % PR#20
\label{sec:georef_tin_3}
Under review.%\markdownInput{../Georeferencing-Tin-12d-3/Readme.md}
\clearpage

\subsection{Georeferencing and DTM 4} % PR#21
\label{sec:georef_tin_4}
Under review.%\markdownInput{../Georeferencing-Tin-12d-4/Readme.md}
\clearpage

% -------------------------------------------------------------------------------
\section{Alignment}

\subsection{Alignment 1}
\label{sec:align_1}
\markdownInput{../Alignment-12d-1/readme.md}
\clearpage

\subsection{Alignment 2}
\label{sec:align_2}
\markdownInput{../Alignment-12d-2/readme.md}
\clearpage

\subsection{Alignment 3}
\label{sec:align_3}
\markdownInput{../Alignment-12d-3/readme.md}
\clearpage

\subsection{Alignment 4}
\label{sec:align_4}
\markdownInput{../Alignment-12d-4/readme.md}
\clearpage

\subsection{Alignment 5}
\label{sec:align_5}
\markdownInput{../Alignment-12d-5/readme.md}
\clearpage

\subsection{Alignment 6} % PR#25
\label{sec:align_6}
Under review.%\markdownInput{../Alignment-12d-6/readme.md}
\clearpage

\subsection{Alignment 7} % PR#26
\label{sec:align_7}
Under review.%\markdownInput{../Alignment-12d-7/readme.md}
\clearpage

\subsection{Alignment 8} % PR#29
\label{sec:align_8}
Under review.%\markdownInput{../Alignment-12d-8/readme.md}
\clearpage

\subsection{Alignment 9} % PR#31
\label{sec:align_9}
Under review.%\markdownInput{../Alignment-12d-9/readme.md}
\clearpage

\subsection{Alignment 10} % PR#38
\label{sec:align_10}
Under review.%\markdownInput{../Alignment-12d-10/readme.md}
\clearpage

\subsection{Alignment 11} % PR#42
\label{sec:align_11}
Under review.%\markdownInput{../Alignment-12d-11/readme.md}
\clearpage

\subsection{Alignment 12} % PR#96
\label{sec:align_12}
Under review.%\markdownInput{../Alignment-Aplitop-1/readme.md}
\clearpage

\subsection{Alignment 13} % PR#39
\label{sec:align_13}
Under review.%\markdownInput{../UT-Alignment-3/readme.md}
\clearpage

\subsection{Alignment 14} % PR#71 *
\label{sec:align_14}
Under review.%\markdownInput{../Alignment-INDOT/readme.md}
\clearpage

\subsection{Alignment 15} % PR#96
\label{sec:align_15}
Under review.%\markdownInput{../Alignment-Aplitop-2/readme.md}
\clearpage

\subsection{Alignment 16} % PR#96
\label{sec:align_16}
Under review.%\markdownInput{../Alignment-Aplitop-3/readme.md}
\clearpage

\subsection{Alignment 17} % PR#103
\label{sec:align_17}
Under review.%\markdownInput{../Georeferencing-Alignment-12d-1/readme.md}
\clearpage

\subsection{Alignment 18} % PR#107
\label{sec:align_18}
Under review.%\markdownInput{../Georeferencing-Alignment-12d-2/readme.md}
\clearpage

% -------------------------------------------------------------------------------
\section{Spatial Structure}

\subsection{Spatial Structure 1}
\label{sec:spatial_1}
\markdownInput{../SpatialStructure-1/readme.md}
\clearpage

\subsection{Spatial Structure 2}
\label{sec:spatial_2}
\markdownInput{../SpatialStructure-2/readme.md}
\clearpage

\subsection{Spatial Structure 3}
\label{sec:spatial_3}
\markdownInput{../SpatialStructure-3/readme.md}
\clearpage

\subsection{Spatial Structure 4}
\label{sec:spatial_4}
\markdownInput{../SpatialStructure-4/readme.md}
\clearpage

\subsection{Spatial Structure 5}
\label{sec:spatial_5}
\markdownInput{../SpatialStructure-5/readme.md}
\clearpage

\subsection{Spatial Structure 6}
\label{sec:spatial_6}
\markdownInput{../SpatialStructure-6/readme.md}
\clearpage

\subsection{Spatial Structure 7} % PR#116 *
\label{sec:spatial_7}
Under review.%\markdownInput{../SpatialStructure-7/README.md}
\clearpage

\subsection{Spatial Structure 8} % PR#65
\label{sec:spatial_8}
Under review.%\markdownInput{../CrossDisciplinary-1/README.md}
\clearpage

% -------------------------------------------------------------------------------
\section{Domain Specifics}

\subsection{Road 1}  % PR#52
\label{sec:road_1}
Under review.%\markdownInput{../UT-IfcCourse/readme.md}
\clearpage

\subsection{Road 2} % PR#52
\label{sec:road_2}
Under review.%\markdownInput{../UT-IfcPavement/readme.md}
\clearpage

\subsection{Road 3} % PR#40
\label{sec:road_3}
Under review.%\markdownInput{../UT-Road-1/readme.md}
\clearpage

\subsection{Road 4} % PR#71 *
\label{sec:road_4}
Under review.%\markdownInput{../SectionedSurface-INDOT/readme.md}
\clearpage

\subsection{Road 5} % PR#88 *
\label{sec:road_5}
Under review.%\markdownInput{../RumbleStrip-INDOT/readme.md}
\clearpage

\subsection{Road 6} % PR#106 *
\label{sec:road_6}
Under review.%\markdownInput{../SectionedSurface-Aplitop-1/readme.md}
\clearpage

\subsection{Bridge 1} % PR#51
\label{sec:bridge_1}
Under review.%\markdownInput{../SectionedSolidHorizontal-1/readme.md}
\clearpage

\subsection{Marine 1} % PR#94
\label{sec:marine_1}
Under review.%\markdownInput{../MarineFurniture-1/readme.md}
\clearpage

\subsection{Marine 2} % PR#97
\label{sec:marine_2}
Under review.%\markdownInput{../MarineFurniture-2/readme.md}
\clearpage

% -------------------------------------------------------------------------------
\section{Earthworks}

\subsection{Earthworks 0}
\label{sec:earth_0}
\markdownInput{../Earthworks-12d-0/readme.md}
\clearpage

\subsection{Earthworks 1}
\label{sec:earth_1}
\markdownInput{../Earthworks-12d-1/readme.md}
\clearpage

\subsection{Earthworks 2}
\label{sec:earth_2}
\markdownInput{../Earthworks-12d-2/readme.md}
\clearpage

\subsection{Earthworks 3}
\label{sec:earth_3}
Under review.%\markdownInput{../UT-Earthworks-01/readme.md}
\clearpage

% -------------------------------------------------------------------------------
\section{Geotechnics}

\subsection{Borehole 1}
\label{sec:borehole_1}
\markdownInput{../Borehole-1/readme.md}
\clearpage

% -------------------------------------------------------------------------------
\section{Geodetic Survey}

\subsection{Survey 1}
\label{sec:survey_1}
\markdownInput{../Georeferencing-Survey-12d-1/readme.md}
\clearpage

\subsection{Survey 2} % PR#56
\label{sec:survey_2}
Under review.%\markdownInput{../Georeferencing-Survey-12d-2/readme.md}
\clearpage

\subsection{Survey 3} % PR#112
\label{sec:survey_3}
Under review.%\markdownInput{../Georeferencing-Survey-12d-3/readme.md}
\clearpage


% -------------------------------------------------------------------------------
\section{Drainage System}

\subsection{Drainage System 1}
\label{sec:drainage_1}
\markdownInput{../DrainageSystem-1/Readme.md}
\clearpage

\subsection{Drainage System 2}
\label{sec:drainage_2}
\markdownInput{../DrainageSystem-2/Readme.md}
\clearpage

% -------------------------------------------------------------------------------
\section{Combinations}

\subsection{Combination 1: Georeferencing and Alignment and DTM} % PR#43
\label{sec:align_dtm_1}
Under review.%\markdownInput{../Georeferencing-Alignment-Tin-12d-1/readme.md}
\clearpage

\subsection{Combination 2: Georeferencing and Earthworks}
\label{sec:georef_earth_1}
\markdownInput{../Georeferencing-Earthworks-12d-1/readme.md}
\clearpage

\subsection{Combination 3: Georeferencing and Earthworks 2} % PR#50
\label{sec:georef_earth_2}
Under review.%\markdownInput{../Georeferencing-Earthworks-12d-2/readme.md}
\clearpage

\subsection{Combination 4: Alignment and DTM}
\label{sec:align_dtm_2}
Under review.%\markdownInput{../Alignment-Tin-1/readme.md}
\clearpage

% -------------------------------------------------------------------------------
\section{Other}

\subsection{Other 1: Cost items} % PR#53 *
\label{sec:other_1}
Under review.%\markdownInput{../UT-CostItem-COWI/readme.md}
\clearpage


\end{document}
